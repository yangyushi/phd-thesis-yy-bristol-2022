\documentclass[11pt,twoside]{report}
\usepackage{preamble}
\begin{document}

\cleardoublepage
\chapter*{Abstract}

A group of animals is a typical active matter system, whose behaviour can be understood from the perspective of statistical physics.

Here we studied the behaviour of zebrafish experimentally. We developed a novel algorithm to extract fish locations from 2D images. By incorporating images captured from synchronised cameras, we calculated the 3D locations of individual fish in a group. The data-processing pipeline developed in this work allows the trajectories of the fish being recovered from the coordinates.
We then analysed the collective behaviour of the zebrafish by different correlation functions. We find the macroscopic states of the fish can switch between ordered and disordered.
The changing states of the fish is dominated by a dynamical length scale as well as a structural length scale.
More specifically, the polarisation of fifty zebrafish correlate robustly with a dimensionless number, the ratio between the persistence length and the nearest neighbour distance of individual fish.

To understand the observed zebrafish behaviour, we proposed different models to fit the density distribution of the fish as well as the dynamics of the fish.
For the spatial distribution of the fish, our model highlights the fish-environment interaction, that dominates the behaviour of small groups ($N < 5$), and the fish-fish interaction, that dominates the behaviour of large groups ($N = 50$). 
For the dynamics of the fish, our model revealed the importance of the orientational inertia for the fish individuals, and the alignment interaction between the fish.

Finally, we applied the established methods to biology, and studied the behavioural feature of the {\mf}, whose genetic modification is related to human diseases. The {\mf} individuals exhibited slower re-orientation, corresponding to a higher activity. Consequently, a group of {\mf} exhibited more ordered behaviour, compared to the wildtype zebrafish. The linkage between the individual behaviour and the collective behaviour of the {\mf} can be explained by our proposed active matter model.


\addcontentsline{toc}{chapter}{Abstract}

\cleardoublepage
\chapter*{}
\vspace*{0.1\textheight}

{\CJKfamily{Ling}\selectfont 
\Large{天涯呀~海角~~觅呀~觅知音}
}

\cleardoublepage

%\vspace*{0.2\textheight}


\cleardoublepage

\chapter*{Acknowledgements}
\addcontentsline{toc}{chapter}{Acknowledgements}

I would like to thank Paddy Royall, Chrissy Hammond, John Russo, and Thomas Machon for helpful guidances on the project. I will not be able to carry out my researches  without the financial support from China Scholar Council and university of Bristol.


I could not thank Francesco Turci, Erika Kague, and Abraham Mauleon Amieva enough for their tremendous supports, as my tutors and my friends. Ioatzin Rios de Anda told me I will finish my study, just by \emph{not giving up}. These are the reasons I did not quit my PhD.


Working in G39 is a pleasant experience. I wish Fergus Moore a nice postdoc in Rome, and Levke Ortlieb to solve the glass transition problem \emph{again}. I hope Jingwen Li (李静文) to find a good job in academia, and Xioayue Wu (武晓岳) to have more fun with coding.

I appreciate the nice chats with Jun Dong (董君), Rui Cheng (程睿), and Laurent Vaughan, the free cigarettes from Maximilian Kloucek and Nariaki Sakaï, the crochet techniques from My Nguyen, and the \emph{friendship} offered by Marcos Villeda Hernandez and Teodoro Garcia Milan.

The Bristol Centre for Functional Nanomaterials (BCFN) provided tremendous amount of help for me to study and live in a foreign country. The management team of BCFN as well as the 2017 cohorts are the best. 
The support from the fish facility in Bristol, as well as the group members in Hammond Lab are extremely helpful. All the talks in Chrissy's group meeting surprised me with the power of biology.


Shamelessly, I copied a lot of code snippets and ideas from the nameless heroes and heroins on the Internet. The charitable contributions from Francesco Turci (\code{@Fturci}), Joshua Robinson (\code{@tranqui}), Peter Crowther (\code{@marrygoat}), Katherine Skipper (\code{@kskips}), and Michael P Allen (\code{@Michael-P-Allen}) were extremely helpful.

% 陈睿 (Rui Cheng); Katie; Wahab; Jun;
% My; Alessandro;

I will not forget the accompany of Min Wang (王敏), and all the sleepless nights we spent together, chatting, worrying, and laughing.

\vfill


\cleardoublepage
\chapter*{Author's declaration}
\addcontentsline{toc}{chapter}{Author's declaration}

I declare that the work in this dissertation was carried out in accordance with the requirements of the University's Regulations and Code of Practice for Research Degree Programmes and that it has not been submitted for any other academic award.
Except where indicated by specific reference in the text, the work is the candidate's own work.
Work done in collaboration with, or with the assistance of, others, is indicated as such. Any views expressed in the dissertation are those of the author.

\vspace{1cm}
\vspace{-3pt}

\hspace{2.5cm} {{\CJKfamily{Kai}\selectfont \Huge{杨雨诗}}} \hspace{5.5cm} 29/03/22 \vspace{-\baselineskip} \vspace{3pt} \\
SIGNED: .............................................................
\qquad
DATE: ..........................

\vspace{1cm}

\section*{Publications:}

\begin{enumerate}
	\item \fullcite{yang2021pcb}
\end{enumerate}

\section*{Other Contributions}

\begin{enumerate}
	\item \fullcite{kague2021}
	\item \fullcite{lopez-cuevas2021}
	\item \fullcite{salazar-silva2021}
\end{enumerate}

\end{document}