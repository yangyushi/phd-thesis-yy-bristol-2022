\documentclass[11pt,twoside]{report}
\usepackage{preamble}
\graphicspath{{../img/ch2/}}
\setcounter{chapter}{1}


\begin{document}


\chapter{The Collective Behaviours of Active Matter Systems}

\label{chapter:collective_behaviour}

\epigraph{We declare that the splendour of the world has been enriched by a new beauty: the beauty of speed.}{F. T. Marinetti, \emph{Manifesto of Futurism}, (1909)}

This chapter would introduce the readers the background of non-equilibrium collective behaviours. Mainly this chapter will be about the collective behaviours of \emph{animals}.

\section{Facets of Collective Behaviours}

\subsection{Active Matter Physics}

\subsection{Understanding Animals}

\subsection{Controlling Robots}


\section{Observations and Experiments}

Experimentalists observed different types of collective behaviours. In this section I will categories these phenomena according to the spatial dimensions. This way of classification stress the technical part of the experiment, since very different methods were required for different dimensions.

\subsection{Collective Behaviours in 1D}

\subsection{Collective Behaviours in 2D}

The movement of fish had interested researchers, and people had been embedding the idea from condensed--mater physics in the fish study since 1978. \cite{}

\subsection{Collective Behaviours in 3D}

\subsection{Collective Behaviours other Dimensions}

The collective behaviours of animals can happen without space. One example is the sound of fogs.


\section{Modelling The collective Behaviours}

\label{ch2-review-model}


\subsection{The Vicsek Model}

\subsubsection{The original Vicsek Model}

\subsubsection{The Inertial Spin Model}

The inertial spin model is another important variant of the Vicsek model, where the dynamic of the system is given by the following Hamiltonian.


\subsection{Vectorial Network Model}

A well studied model is the 2D Vectorial Network Model (VNM). Sharing the same velocity updating rule with the Vicsek model, the VNM used a graph to determine the neighbours between particles, instead of determining the neighbours based on the spatial relationships. 

The disordered states are composed of purely randomised vectors. The order parameter, noted as the polarisation $\Phi$, is constantly 0.

\citeauthor{aldana2003} analysed the model in detail, and claimed the continues phase transition in VNM can be described by equation \cite{aldana2003},

\begin{equation}
	\Phi=\left\{\begin{array}{ll}{\left[C\left(\eta_{c}-\eta\right)\right]^{1 / 2}} & \text { for } 0<\eta_{c}-\eta \ll 1 \\ 0 & \text { for } \eta>\eta_{c}\end{array}\right.	
\end{equation}

\noindent where $\Phi$ is the polarisation order parameter, $\eta$ is the noise level, and $\eta_c$ is the critical noise value. In addition, the relationship between the polarisation ($\Phi$) and $\eta$, near the critical point, is approximated by

\begin{equation}
\Phi \approx \sqrt{
	\frac{2(\sin \eta-2 \eta)}{(K-2) \sin \eta-(K-3) \eta}\left(1-\frac{\eta}{\sqrt{\pi K} \sin (\eta / 2)}\right)
},
\end{equation}

\noindent where $K$ is the number of neighbours of the particles. The result matches the numerical results. This analytical expression covered the ``critical'' region of the model, and can not be extrapolated to the ordered states.

Years after \citeauthor{aldana2003}'s work, \citeauthor{porfiri2014} derived the expression for the ordered states VNM, which writes as \cite{porfiri2014}

\begin{equation}
\Phi \approx 1-\eta^{2} \frac{\pi^{2}}{6} \frac{K(N-1)}{N(K-1)+1},
\end{equation}

\noindent where $N$ is the number of particles.


\subsection{Other Models}

\subsubsection{The Active Brownian Particles}

It is worth mentioning another, often noted as the active Brownian particles (ABP). This model is often used to describe the behaviour of colloids, micro particles suspended in density--matched solvents.


\end{document}
