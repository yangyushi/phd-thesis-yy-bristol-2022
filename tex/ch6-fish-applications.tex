\documentclass[11pt,twoside]{report}
\usepackage{preamble}
\graphicspath{{../img/}}
\setcounter{chapter}{6}

\begin{document}

\chapter{Behavioural Changes of Zebrafish with Biological Perturbations}

\epigraph{子非魚\; 安知魚之樂\\[1ex]You are not a fish, how could you know about the happiness of the fish}{莊子}

\section{Introduction}

Understanding the behaviours of zebrafish is not only intellectually fulfilling, but it is also useful. Appreciating the stochastic nature of the animal behaviours, the detailed knowledge of zebrafish would enable us identify the behavioural differences among different groups, without being mislead by the randomness of animal behaviours.

Let us take an example from soft matter physics. If I create two canonical ensembles of Lennard--Jones liquid at the same density and temperature (the same $N, V, T$ parameters), and follow the movement of particles, I would get different trajectories, but same thermodynamic quantities, such as the pressure and the entropy, since the systems are equivalent. If these ensembles have different temperature values, I would get wildly different results, even different phases, because the systems are in different states. Nevertheless, I can still confirm these systems are the same if they follow the same equation of states.

I will now extend the argument to the zebrafish. Suppose I have two groups of zebrafish, and they have different statistical quantities, such as the averaged speed and the averaged nearest neighbour (NN) distance, what conclusion should I draw? I have two options. Firstly I can claim these two groups of fish are inherently different, which means they interact with each other in different ways. Alternatively, I can claim that these two groups of fish are in different states, while they have the same intrinsic fish--fish interactions. I can use the equation of states to determine the better conclusion. If the different groups follow the same equation of states, then I am confident that they have the same interaction but were subjection parameters (like LJ system under different temperatures). In another case, I can conclude that the fish are intrinsically different with different fish--fish interactions.

In this chapter, I will use the methods developed before, and study the behavioural change of zebrafish when they were genetically modified. Following a pioneering work \cite{tang2020} that tested the behaviours of different fish groups with different genetic modifications, I will present a more detailed analysis of the behaviours with a specific mutation, for which the \emph{col11a2} gene was changed.

\section{Method}

\section{Results}

\printbibliography

\end{document}
